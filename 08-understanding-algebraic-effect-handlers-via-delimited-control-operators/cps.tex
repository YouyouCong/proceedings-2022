\newcommand{\LambdaszCPS}{
\begin{figure}[t]

\begin{align*}
\cpsv{x} &= x \\
\cpsv{\abs{x}{M}} &= \abs{x}{\cps{M}} \\
\cps{\ret{V}} &= \abs{k}{\app{k}{\cpsv{V}}} \\
\cps{\app{V}{W}} &= \app{\cpsv{V}}{\cpsv{W}} \\
\cps{\letin{x}{M}{N}} &= \abs{k}{\app{\cps{M}}{(\abs{x}{\app{\cps{N}}{k}})}} \\
\cps{\shiftz{k}{M}} &= \abs{k}{\cps{M}} \\
\cps{\dollarret{M}{x}{N}} &= \app{\cps{M}}{(\abs{x}{\cps{N}})}
\end{align*}

\caption{CPS Translation of \lambdasz~\cite{hillerstrom-cps,materzok-subtyping}}
\label{fig:lambdaszcps}
\end{figure}
}

\newcommand{\LambdahCPS}{
\begin{figure}[t]

\begin{align*}
\cps{\doop{V}} &=
  \abs{k}{\abs{h}{\apptwo{h}{\cpsv{V}}{(\abs{x}{\apptwo{k}{x}{h}})}}} \\
\cps{\handle{M}{x}{M_r}{x}{k}{M_h}} &=
  \apptwo{\cps{M}}{(\abstwo{x}{h}{\cps{M_r}})}{(\abstwo{x}{k}{\cps{M_h}})}
\end{align*}

\caption{CPS Translation of \lambdah~\cite{hillerstrom-cps}
(cases for $\lambda$-terms are the same as Figure~\ref{fig:lambdaszcps})}
\label{fig:lambdahcps}
\end{figure}
}

\newcommand{\Macro}{
\begin{figure}[t]

\begin{align*}
\macro{\doop{V}} &=
  \shiftz{k}
         {\ret{(\abs{h}
                    {\letnoin{v_1}{\app{h}{\macro{V}}}}}} \\
& \hspace{2.88cm}    \intt \ v_1 \ 
                             (\abs{x}
                                  {\letnoin{v_2}{\app{k}{x}}} \\
& \hspace{4.45cm}                   \intt \ \app{v_2}{h})) \\
\macro{\handle{M}{x}{M_r}{x}{k}{M_h}} &=
  \letnoin{v}{\dollarret{\macro{M}}{x}{\ret{\abs{h}{\macro{M_r}}}}} \\
& \hspace{4.25mm} \intt \ \app{v}{(\abs{x}{\ret{(\abs{k}{\macro{M_h}})}})}
\end{align*}

\caption{Macro Translation from \lambdah to \lambdasz}
\label{fig:macro}
\end{figure}
}

\newcommand{\SMacro}{
\begin{figure}[t]

\begin{align*}
\macro{\doop{V}} &=
  \shiftz{k}
         {(\abs{h}
               {\apptwo{h}{\macro{V}}{(\abs{x}{\apptwo{k}{x}{h}})}})} \\
\macro{\handle{M}{x}{M_r}{x}{k}{M_h}} &=
  \app{\dollarret{\macro{M}}{x}{\abs{h}{\macro{M_r}}}}
      {(\abstwo{x}{k}{\macro{M_h}})}
\end{align*}

\caption{Simpler Macro Translation from \lambdah to \lambdasz~\cite{forster-jfp}}
\label{fig:smacro}
\end{figure}
}

\section{Deriving a CPS Translation}
\label{sec:cps}

The classical way of specifying the semantics of control operators is to
give a translation into continuation-passing style (CPS)~\cite{plotkin-cps},
which converts control operators into plain lambda terms.
In the case of \shiftztt and dollar (or reset), there exist several variants
of CPS translation, differing in the representation of
continuations~\cite{biernacki-toplas,dyvbig-monadic,shan-simulation,materzok-subtyping,materzok-hierarchy}.

Compared to control operators, the semantics of effect handlers seems to
be less tightly tied to the CPS translation.
In fact, the CPS translation of effect handlers has not been formally
studied until recently~\cite{hillerstrom-cps,hillerstrom-jfp}.
This gives rise to a question: What would we obtain if we derive the CPS
translation of effect handlers from that of control operators, which is
considered definitional?

In this section, we derive the CPS translation of effect handlers by
composing the following translations.

\begin{enumerate}
\item The macro translation from effect handlers to
  \shiftztt/dollar~\cite{forster-jfp}
\item The CPS translation of \shiftztt/dollar~\cite{materzok-hierarchy}
\end{enumerate}

\noindent We show that, by avoiding ``administrative'' constructs in the 
macro translation, and by being selective in the CPS translation, we obtain 
the same CPS translation as the unoptimized\footnote{By ``unoptimized 
translation'', we mean the first-order translation in Figure 5 of 
Hillerstr\"om et al.~\cite{hillerstrom-cps}.}
translation given by Hillerstr\"om et al.~\cite{hillerstrom-cps}.
In what follows, we review the existing CPS translations of \shiftztt/dollar
and effect handlers (Sections~\ref{sec:cps:shiftz} and \ref{sec:cps:handler}),
as well as the macro translation from effect handlers to \shiftztt/dollar
(Section~\ref{sec:cps:macro}).
We then compose the first and third translations to obtain the second
translation (Section~\ref{sec:cps:compose}), thus formally relating the CPS
translations of \shiftztt/dollar and effect handlers.


\subsection{CPS Translation of Shift0 and Dollar}
\label{sec:cps:shiftz}

In Figure~\ref{fig:lambdaszcps}, we present the CPS translation from
\lambdasz to the plain $\lambda$-calculus (with no value vs. computation 
distinction).
The definition is adopted from Hillerstr\"om et al.~\cite{hillerstrom-cps}
(for the $\lambda$-terms fragment) and Materzok and
Biernacki~\cite{materzok-hierarchy} (for \shiftztt/dollar).
We have two mutually-defined translations $\cpsv{\_}$ and $\cps{\_}$,
taking care of values and computations, respectively.
The former yields a direct-style value in the $\lambda$-calculus, whereas
the latter yields a continuation-taking function.
To go through the cases of the control operators, the translation turns a
\shiftztt construct into a $\lambda$-abstraction, and a dollar construct
into an application of the main computation $\cps{M}$ to an ending
continuation $\abs{x}{\cps{N}}$\footnote{The CPS translation of the original
dollar operator $\dollar{N}{M}$ is defined as
$\abs{k}{\app{\cps{N}}{(\abs{f}{\apptwo{\cps{M}}{f}{k}})}}$.}.

\LambdaszCPS


\subsection{CPS Translation of Effect Handlers}
\label{sec:cps:handler}

\LambdahCPS

In Figure~\ref{fig:lambdahcps}, we present the CPS translation from
\lambdah to the plain $\lambda$-calculus.
The definition is adopted from Hillerstr\"om et al.~\cite{hillerstrom-cps}.
We again have separate translations for values and computations.
Among them, the computation translation yields a function that takes in two
continuations: a pure continuation $k$ for returning a value, and an effect
continuation $h$ for handling an operation.
To go through the interesting cases, the translation of an operation
calls the effect continuation $h$ representing the interpretation of that
operation\footnote{The original translation of
Hillerstr\"om et al.~\cite{hillerstrom-cps} packages the two arguments to
$h$ into a tuple and associates the tuple with an operation label.}.
Notice that the resumption $\abs{x}{\apptwo{k}{x}{h}}$ passed to $h$
includes $h$ itself, reflecting the deep nature of handlers.
The translation of a handler sets the two continuations for the
handled computation $\cps{M}$.
Other cases are the same as the translation of \lambdasz.
In particular, the translation of the \returntt and \lettt expressions 
requires only one continuation argument, as the effect continuation can 
be removed by $\eta$-reduction.


\subsection{Macro Translation from \lambdah to \lambdasz}
\label{sec:cps:macro}

\Macro

\SMacro

Having seen the CPS translations, we look at the macro translation from
\lambdah to \lambdasz (Figure~\ref{fig:macro}), which is adapted from
Forster et al.~\cite{forster-jfp}.
Roughly speaking, the translation converts operations into \shiftztt
and handlers into dollar.
Technically, it introduces an effect-continuation-passing mechanism to
make the operation clause accessible to the \shiftztt operator.
The result of the translation is somewhat verbose due to the presence of
the \returntt and \lettt expressions; these are necessary for ensuring that
the translation produces a well-formed expression in fine-grain call-by-value.
The original translation of Forster et al. (Figure~\ref{fig:smacro}) is
simpler, as it is defined on call-by-push-value (where the operator of an
application may be a computation).
Note that, in both versions, the translation of $\lambda$-terms is defined
homomorphically.


\subsection{Composing Macro and CPS Translations}
\label{sec:cps:compose}

Now, let us derive a CPS translation of \lambdah by composing the macro
translation on \lambdah and the CPS translation on \lambdasz.

\paragraph{Operations}

We begin with the case of operations.
Below is the na\"ive composition of the macro and CPS translations.

\vspace{-3mm}

\begin{align*}
& \hspace{4.7mm} \cps{\macro{\doop{V}}} \\
% macro
&= \cps{\shiftz{k}
               {\ret{(\abs{h}
                          {\letin{v_1}{\app{h}{\macro{V}}}
                                 {\app{v_1}
                                      {(\abs{x}
                                            {\letin{v_2}{\app{k}{x}}
                                                   {\app{v_2}{h}}})}}}}}} \\
% CPS
&= \abs{k}
       {\abs{k_1}
            {\app{k_1}
                 {(\abs{h}
                       {\abs{k_2}
                            {\app{\app{h}{\cpsv{\macro{V}}}}
                                 {(\abs{v_1}
                                       {\apptwo{v_1}
                                               {(\abs{x}
                                                     {\abs{k_3}
                                                          {\app{\app{k}{x}}
                                                               {(\abs{v_2}{\apptwo{v_2}{h}{k_3}})}}})}
                                               {k_2}})}}})}}}
\end{align*}

\vspace{1mm}

\noindent The result has four continuation variables: $k$, $k_1$, $k_2$,
and $k_3$.
Among them, the latter three continuations come from the \returntt and
\lettt expressions introduced by the macro translation.
As we mentioned before, these \returntt and \lettt are only necessary for
the well-formedness of macro-translated expressions.
In other words, they do not have any computational content.
What this implies is that, when our ultimate goal is to convert effect
handlers into plain $\lambda$-terms, we can use a simpler macro translation
that does not yield these administrative \returntt and \lettt.
This simpler translation is exactly the original translation of Forster et
al.~\cite{forster-jfp}.

\vspace{-3mm}

\begin{align*}
& \hspace{4.7mm} \cps{\macro{\doop{V}}} \\
% simpler macro
&= \cps{\shiftz{k}
               {\abs{h}
                    {\apptwo{h}{\macro{V}}{(\abs{x}{\apptwo{k}{x}{h}})}}}}
\end{align*}

\vspace{1mm}

\noindent The result of the simpler macro translation is not a proper 
expression in fine-grain CBV: the \shiftztt operator has a value body,
and the body has applications of a function to two values.
This prevents us from directly applying the CPS translation in 
Figure~\ref{fig:lambdaszcps}.
However, we can still reuse the CPS translation, under the principle
``do not introduce a continuation when it is not necessary''. 
In the case of the above expression, we simply apply the value 
translation to the body of \shiftztt and the arguments of the
applications.
Such a principle reminds us of \textit{selective} CPS translations in 
the literature~\cite{nielsen-selective,rompf-selective,asai-selective}.

\vspace{-3mm}

\begin{align*}
& \hspace{4.7mm} \cps{\shiftz{k}
                 {\abs{h}
                      {\apptwo{h}{\macro{V}}{(\abs{x}{\apptwo{k}{x}{h}})}}}} \\
  % CPS
&= \abstwo{k}{h}{\apptwo{h}{\cpsv{\macro{V}}}{(\abs{x}{\apptwo{k}{x}{h}})}}
\end{align*}

\vspace{1mm}

\noindent The result of composition is identical to the existing
translation of operations, which we saw in Figure~\ref{fig:lambdahcps}.

\paragraph{Handlers}

We next look at the handler case.
Below is the na\"ive composition of the two translations.

\vspace{-3mm}

\begin{align*}
& \hspace{4.7mm} \cps{\macro{\handle{M}{x}{M_r}{x}{k}{M_h}}} \\
% macro
&= \cps{\letin{v}{\dollarret{\macro{M}}{x}{\ret{\abs{h}{\macro{M_r}}}}}
                 {\app{v}{(\abs{x}{\ret{\abs{k}{\macro{M_h}}}})}}} \\
% CPS
&= \abs{k_1}
       {(\app{\cps{\macro{M}}}
             {(\abs{x}{\abs{k_2}{\app{k_2}{(\abs{h}{\cps{\macro{M_r}}})}}})})} \\
& \hspace{1.16cm} (\abs{v}
                      {\app{\app{v}{(\abs{x}{\abs{k_3}{\app{k_3}{(\abs{k}{\cps{\macro{M_h}}})}}})}}
                           {k_1}})
\end{align*}

\vspace{1mm}

\noindent As in the case of operations, the result has three continuation
variables $k_1$, $k_2$, and $k_3$ that originate from the macro translation.
To avoid these continuation variables, we first apply the simpler macro 
translation, and then selectively apply the CPS traslation.
This makes the composition more concise, as shown below.

\vspace{-3mm}

\begin{align*}
& \hspace{4.7mm} \cps{\macro{\handle{M}{x}{M_r}{x}{k}{M_h}}} \\
% simpler macro
&= \cps{\app{\dollarret{\macro{M}}{x}{\abs{h}{\macro{M_r}}}}
            {(\abs{x}{\abs{k}{\macro{M_h}}})}} \\
% CPS
&= \app{\cps{\macro{M}}{(\abstwo{x}{h}{\cps{\macro{M_r}}})}}
       {(\abstwo{x}{k}{\cps{\macro{M_h}}})}
\end{align*}

\vspace{1mm}

\noindent The result is again identical to the existing translation 
of handlers, which we saw in Figure~\ref{fig:lambdahcps}.
