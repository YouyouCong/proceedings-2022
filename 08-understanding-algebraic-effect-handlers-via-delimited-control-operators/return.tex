\newcommand{\Lambdahn}{
\begin{figure}[t]

\lefttext{Syntax}
\begin{align*}
V, W &::= x \mid \abs{x}{M} \mid \shade{\pair{l}{V}} && \text{Values} \\
M, N &::= \ret{V} \mid \app{V}{V} \mid \letin{x}{M}{M} \mid \doop{V}
     && \text{Computations} \\
     & \mid \shade{\casep{V}{l}{x}{M}} \mid \shade{\casel{l}{M}{M}} \\
     & \mid \handleone{M}{x}{k}{M} \\
l &::= \shade{\optt} \mid \shade{\rettt} && \text{Labels}
\end{align*}

\medskip

\lefttext{Reduction}
\begin{align*}
E[\app{(\abs{x}{M})}{V}]
  &\reduce E[\subst{M}{x}{V}]
  \hspace{3.58cm} \rulename{$\beta_v$} \\
E[\letin{x}{\ret{V}}{M}]
  &\reduce E[\subst{M}{x}{V}]
  \hspace{3.58cm} \rulename{$\zeta_v$} \\
E[\casep{\pair{l}{V}}{\lpr}{x}{M_r}]
  &\reduce E[\substtwo{M_r}{\lpr}{l}{x}{V}]
  \hspace{2.86cm} \rulename{$\iota_p$} \\
E[\casel{\rettt}{M_r}{M_h}]
  &\reduce E[M_r]
  \hspace{4.23cm} \rulename{$\iota_{\mathit{ret}}$} \\
E[\casel{\optt}{M_r}{M_h}]
  &\reduce E[M_h]
  \hspace{4.21cm} \rulename{$\iota_{\mathit{op}}$} \\
E[\handleone{\ret{V}}{x}{k}{M_h}]
  &\reduce E[\ret{V}]
  \hspace{3.31cm} \rulename{$\beta_h$} \\
E[\handleone{F[\doop{V}]}{x}{k}{M_h}]
  &\reduce E[\substtwo{M}{x}{V}{k}{f}]
  \hspace{2.93cm} \rulename{$\beta_{do}$} \\
  & \hspace{5.2mm} \text{where } f = \abs{y}{\handleone{F[y]}{x}{k}{M_h}}
\end{align*}

\caption{\lambdahn: A Calculus of Effect Handlers without the Return Clause}
\label{fig:lambdahn}
\end{figure}
}

\section{Adding and Removing the Return Clause}
\label{sec:return}

The dollar construct $\dollarret{M}{x}{N}$ in \lambdasz is a
generalization of the ``reset'' construct $\resetz{M}$, which is more
commonly found in the continuations literature.
The reset construct does not have an ending continuation; it simply
evaluates the body $M$ in an empty context.
As shown by Materzok and Biernacki~\cite{materzok-axiom}, the dollar and
reset constructs can macro-express~\cite{felleisen-macro} each other.
That is, there is a pair of local translations, called \textit{macro
translations}, that add and remove the ending continuation while preserving
the meaning of the program.

It is easy to see that the return clause of an effect handler plays a
similar role to the ending continuation of dollar.
Thus, we can naturally consider a variant of effect handlers without the
return clause.
Such handlers are not uncommon in
formalizations~\cite{schuster-capability,xie-evidently,zhang-bidirectional}
as they simplify the reduction and typing rules.
Now the reader might wonder: Does the existence of the return clause affects
the expressiveness of an effect handler calculus?

In this section, we define macro translations between handlers with and
without the return clause.
Among different ways of removing the return clause, we take a relatively 
intuitive approach that relies on a labeling mechanism.
In what follows, we review the macro translations between dollar and reset
(Section~\ref{sec:return:dollar}), then adapt the translations to effect
handlers (Section~\ref{sec:return:handler}), and lastly prove the
correctness of the translations (Section~\ref{sec:return:correct}).


\subsection{Translating Between Dollar and Reset0}
\label{sec:return:dollar}

Materzok and Biernacki~\cite{materzok-hierarchy} define the macro
translations $\macro{\_}$ between dollar and \resetztt as
follows\footnote{The macro translation is originally defined as:
\im{
\app{(\abs{k}{\resetz{\app{(\abs{x}{\shiftz{z}{\app{k}{x}}})}{\macro{M}}}})}
    {\macro{N}}
}
We adapted the translation by sequencing the application, and by incorporating
the fact that $N$ is always a $\lambda$-abstraction.}.

\lefttext{From dollar to reset}
\begin{align*}
\macro{\dollarret{M}{x}{N}} &=
\resetz{\letin{y}{\macro{M}}{\shiftz{z}{\app{(\abs{x}{\macro{N}})}{y}}}}
\end{align*}

\lefttext{From reset to dollar}
\begin{align*}
\macro{\resetz{M}} &= \dollarret{\macro{M}}{x}{\ret{x}}
\end{align*}

\noindent The translation from reset to dollar is straightforward: it
simply adds a trivial ending continuation.
The translation from dollar to reset is more involved: it wraps the
computation $\macro{M}$ around a reset, and inserts a \shiftztt to
remove the surrounding reset.
The removal of reset is necessary for preservation of meaning (the return
clause should not be evaluated under the original handler), and is realized
by discarding the captured continuation $z$.

Note that the translation of other constructs is defined homomorphically.
For instance, we have $\macro{\abs{x}{M}} = \abs{x}{\macro{M}}$.


\subsection{Translating Between Handlers with and without Return Clause}
\label{sec:return:handler}

\Lambdahn

Guided by the translations between dollar and reset, we define macro
translations between handlers with and without the return clause.
We can easily imagine that adding the return clause is simple.
To remove the return clause, we must somehow implement the removal of a
handler.
In a calculus of effect handlers, any non-local control is triggered by an
operation call.
This means we need to make an operation call when we wish to remove a
handler.
Unlike \shiftztt, however, an operation call does not have an interpretation
on its own.
This means we need to implement the removing behavior in the surrounding
handler, while distinguishing the ``return operation'' from regular
operations.

In Figure~\ref{fig:lambdahn}, we define a calculus of effect handlers
without the return clause, which we call \lambdahn.
The calculus has labels $l$ and pairs $\pair{l}{V}$, allowing us to represent
the return and regular operations as $\doopl{\rettt}{V}$ and $\doopl{\optt}{V}$,
respectively.
The calculus also has pattern matching constructs, with which we can interpret
the two kinds of operations differently in a handler.
Note that these facilities are introduced only for the translation purpose.
That is, we assume that the user can only program with unlabeled operations;
they do not have access to the shaded constructs in Figure~\ref{fig:lambdahn}.

We now define the macro translations between \lambdah and \lambdahn.

\lefttext{From \lambdah to \lambdahn}
\begin{align*}
\macro{\doop{V}} &= \doopl{\optt}{\macro{V}} \\
\macro{\handle{M}{x}{M_r}{x}{k}{M_h}}
  &= \handlett \ (\letnoin{y}{\macro{M}} \\
  & \hspace{1.88cm} \intt \ \doopl{\rettt}{y}) \\
  & \hspace{4.8mm} \withtt \ \{ p, k \rightarrow \caseptt \ p \ \oftt \ \{ \pair{l}{x} \rightarrow \\
  & \hspace{5.2mm} \{ \casett_l \ l \ \oftt \\
  & \hspace{5.6mm} \{ \rettt \rightarrow \macro{M_r}; \, \optt \rightarrow \macro{M_h} \} \} \} \}
\end{align*}

\lefttext{From user fragment of \lambdahn to \lambdah}
\begin{align*}
\macro{\doop{V}} &= \doop{\macro{V}} \\
\macro{\handleone{M}{x}{k}{M_h}}
  &= \handlett \ \macro{M} \\
  &\hspace{4.8mm} \withtt \ \handler{x}{\ret{x}}{x}{k}{\macro{M_h}}
\end{align*}

\noindent The first translation\footnote{
If we wish to apply these translations to a typed language, we would need to 
slightly modify the definition.
The reason is that, the translation of regular handlers yields a single 
pattern variable $x$ representing the arguments of the \rettt and \optt 
operations, which have different types in general.
}\footnote{An anonymous reviewer suggests a different translation that does 
not use labels:

\vspace{-2mm}

\begin{align*}
&\hspace{4.3mm} \macro{\handle{M}{x}{M_r}{x}{k}{M_h}} \\
&= (\handlett \ (\letin{y}{\macro{M}}{\ret{\abs{\_}{\app{(\abs{x}{\macro{M_r}})}{y}}}}) \\
&\hspace{5.7mm}  \withtt \ \{ x, \kpr. \, \ret{\abs{\_}{\subst{\macro{M_h}}{k}{\abs{z}{\apptwo{\kpr}{z}{0}}}}} \}) \\
&\hspace{4.3mm} 0
\end{align*}

\noindent where $\app{M}{V}$ is a shorthand for $\letin{x}{M}{\app{x}{V}}$
and $0$ can be replaced by any constant.
The translation implements the return clause by creating thunks (to exit from 
a handler) and passing around a dummy argument (to trigger computation).
This eliminates the need for labels, but in our view, it also makes the 
translation slightly harder to understand.
} attaches a label $\optt$ to regular operations
and simulates the return clause by performing a $\rettt$ operation, which
removes the surrounding handler by discarding the continuation $k$.
The second translation is fairly easy.


\subsection{Correctness}
\label{sec:return:correct}

The macro translations defined above preserve the meaning of programs.
We state this property as the following theorem.

\begin{theorem}[Correctness of Macro Translations]
\label{thm:correctmacro}
Let $=$ be the least congruence relation that includes the reduction
$\reduce$ in \lambdah and \lambdahn.

\begin{enumerate}
\item If $M = N$ in \lambdah, then $\macro{M} = \macro{N}$ in \lambdahn.

\item If $M = N$ in the user fragment of \lambdahn (i.e., the fragment
  consisting of non-shaded constructs), then $\macro{M} = \macro{N}$ in
  \lambdah.
\end{enumerate}
\end{theorem}

\begin{proof}
By cases on the reduction relation $M \reduce N$.
%We provide the proof of interesting cases in Appendix~\ref{app:retprf}.
\end{proof}
