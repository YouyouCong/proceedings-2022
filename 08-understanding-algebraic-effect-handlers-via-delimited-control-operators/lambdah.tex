\newcommand{\Lambdah}{
\begin{figure}[t]

\lefttext{Syntax}
\begin{align*}
V, W &::= x \mid \abs{x}{M} && \text{Values} \\
M, N &::= \ret{V} \mid \app{V}{V} \mid \letin{x}{M}{M} && \text{Computations} \\
     &\mid \doop{V} \mid \handle{M}{x}{M}{x}{k}{M}
\end{align*}

\medskip

\lefttext{Evaluation Contexts}
\begin{align*}
E &::= \hole \mid \letin{x}{E}{M} \mid \handle{E}{x}{M}{x}{k}{M} 
  && \text{General Contexts} \\
F &::= \hole \mid \letin{x}{F}{M} 
  && \text{Pure Contexts}
\end{align*}

\medskip

\lefttext{Reduction}
\begin{align*}
E[\app{(\abs{x}{M})}{V}]
  &\reduce E[\subst{M}{x}{V}]
  \hspace{3cm} \rulename{$\beta_v$} \\
E[\letin{x}{\ret{V}}{M}]
  &\reduce E[\subst{M}{x}{V}]
  \hspace{3cm} \rulename{$\zeta_v$} \\
E[\handle{\ret{V}}{x}{M_r}{x}{k}{M_h}]
  &\reduce E[\subst{M_r}{x}{V}]
  \hspace{2.88cm} \rulename{$\beta_h$} \\
E[\handle{F[\doop{V}]}{x}{M_r}{x}{k}{M_h}]
  &\reduce E[\substtwo{M_h}{x}{V}{k}{f}]
  \hspace{2.18cm} \rulename{$\beta_{\mathit{do}}$} \\
  & \hspace{5.5mm}
  \text{where } f = \abs{y}{\handlett \ F[\ret{y}]} \\
  & \hspace{2.64cm} \withtt \ \handler{x}{M_r}{x}{k}{M_h}
\end{align*}

\caption{Syntax and Reduction Rules of \lambdah}
\label{fig:lambdah}
\end{figure}
}

\section{\lambdah: A Calculus of Effect Handlers}
\label{sec:lambdah}

As a calculus of effect handlers, we consider a restricted variant of
Hillerstr\"om et al.'s calculus of deep handlers~\cite{hillerstrom-cps},
which we call \lambdah.
In Figure~\ref{fig:lambdah}, we present the syntax and reduction rules
of \lambdah.
The calculus differs from Hillerstr\"om et al.'s in that it features
unlabeled operations.
This means handlers in \lambdah can only handle a single operation.
The restriction helps us concentrate on the connection to the \lambdasz
calculus.

Among the effect constructs, $\doop{V}$ performs an operation with
argument $V$.
The other construct $\handle{M}{x}{M_r}{x}{k}{M_h}$ computes the main
computation $M$ in a delimited context, and handles the result of $M$
using the return clause $M_r$ and the operation clause $M_h$.

There are again two reduction rules for the effect constructs.
If the main computation of a handler evaluates to a value $V$, the whole
expression evaluates to the return clause $N$ with $V$ bound to $x$
(rule \rulename{$\beta_h$}).
If the main computation evaluates to $F[\doop{V}]$, where $F$ is a pure
evaluation context that has no handler surrounding a hole, the whole expression
evaluates to the operation clause $M_h$, with $x$ being $V$ and $k$ being
the captured continuation (often called ``resumption'' in the effect handlers
literature) $\abs{y}{\handle{F[\ret{y}]}{x}{M_r}{x}{k}{M_h}}$
(rule \rulename{$\beta_{do}$}).
Notice that the continuation includes the handler that was originally
surrounding the operation.
This design is shared with \shiftztt, and characterizes handlers in
\lambdah as deep ones~\cite{kammar-handler}.
Notice next that the operation clause is evaluated without being surrounded
by the original handler.
This is another similarity to \shiftztt, and allows handlers to capture
a metacontext.

\Lambdah
